\documentclass[12pt,letterpaper,oneside]{memoir}

\usepackage{amsfonts}
\usepackage{amsmath}
\usepackage{amsthm}
\usepackage[urldate=iso8601,date=iso8601]{biblatex}
\usepackage{color}
\usepackage{datetime2}
\usepackage[T1]{fontenc}
\usepackage[utf8]{inputenc}
\usepackage{listings}
\usepackage[zerostyle=b,scaled=0.85]{newtxtt}
\usepackage{lipsum}
\usepackage{url}

\usepackage[hidelinks]{hyperref}
\usepackage{cleveref}

\chapterstyle{section}
% New Commands
\newcommand{\code}[1]{\texttt{#1}}
\newcommand{\email}[1]{\href{mailto:#1}{\code{<#1>}}}
\newcommand{\person}[2]{#1 \email{#2}}
\newcommand{\pharpend}{\person{Peter Robert Harpending}{u0284592@utah.edu}}
\newcommand{\answergraph}[1]{\begin{center}\includegraphics[width=0.8\textwidth]{#1}\end{center}}
\newcommand{\N}{\mathbb{N}}
\newcommand{\R}{\mathbb{R}}
\newcommand{\Q}{\mathbb{Q}}
\newcommand{\Z}{\mathbb{Z}}
\newcommand{\scomp}[2]{\left\{\; #1 \,\mid\, #2 \;\right\}}

% New ``theorems''
\theoremstyle{definition}
\newtheorem{example}{Example}[section]

% New Environments
\newenvironment{cpn}
  {\begin{flushleft}\hrulefill\par\setlength{\parskip}{2ex}}
  {\par\hrulefill\end{flushleft}}
\newenvironment{rclmath}
  {\begin{equation}\begin{tabu}{rcl}}
  {\end{tabu}\end{equation}}

% Extraneous setup
\DTMsetstyle{iso}
\addbibresource{assort.bib}
\nocite{*}

% Listings
\definecolor{mygreen}{rgb}{0.3,0.6,0.3}
\definecolor{mygray}{rgb}{0.8,0.8,0.8}
\definecolor{mymauve}{rgb}{0.58,0,0.82}
\lstset{ %
  % backgroundcolor=\color{white},   % choose the background color; you must add \usepackage{color} or \usepackage{xcolor}
  basicstyle=\footnotesize\ttfamily,     % the size of the fonts that are used for the code
  breakatwhitespace=false,         % sets if automatic breaks should only happen at whitespace
  breaklines=false,                 % sets automatic line breaking
  captionpos=b,                    % sets the caption-position to bottom
  commentstyle=\color{mygreen},    % comment style
  deletekeywords={...},            % if you want to delete keywords from the given language
  escapeinside={\%*}{*)},          % if you want to add LaTeX within your code
  extendedchars=true,              % lets you use non-ASCII characters; for 8-bits encodings only, does not work with UTF-8
  frame=single,                    % adds a frame around the code
  keepspaces=true,                 % keeps spaces in text, useful for keeping indentation of code (possibly needs columns=flexible)
  % keywordstyle=\color{blue},       % keyword style
  % Actually, we are using Idris, but Haskell is close enough
  % language=\null,                % the language of the code
  % morekeywords={*,...},            % if you want to add more keywords to the set
  numbers=left,                    % where to put the line-numbers; possible values are (none, left, right)
  numbersep=5pt,                   % how far the line-numbers are from the code
  numberstyle=\footnotesize\ttfamily,    % the style that is used for the line-numbers
  rulecolor=\color{mygray},        % if not set, the frame-color may be changed on line-breaks within not-black text (e.g. comments (green here))
  showspaces=false,                % show spaces everywhere adding particular underscores; it overrides 'showstringspaces'
  showstringspaces=false,          % underline spaces within strings only
  showtabs=false,                  % show tabs within strings adding particular underscores
  stepnumber=1,                    % the step between two line-numbers. If it's 1, each line will be numbered
  % stringstyle=\color{mymauve},     % string literal style
  tabsize=2,                       % sets default tabsize to 2 spaces
  title=\lstname,                   % show the filename of files included with \lstinputlisting; also try caption instead of title
  caption=\lstname                  % show the filename of files included with \lstinputlisting; also try caption instead of title
}

% Actual document
\begin{document}
\title{Models of assortative mating with various parameterizations}
\author{\pharpend}
\date{\Today}
\maketitle

\tableofcontents

\chapter{Introduction}

In its most abstract sense, assortative mating is the idea that, given a pool
$P$ of individuals, some quantifiable trait $t : P \to (0, 1)$, there is a function
$$M : (P \to (0, 1)) \to P \to P \to (0, 1),$$ such that for any individual $I
\in P$, the function $$M_t(I) : P \to (0, 1)$$ gives, for any individual
$J \in P$, the probability that the individual $I$ will choose to mate with $J$
based on the value $t(I)$. \cite{needed} Note that $$M_t : P \to P \to (0, 1)$$
is not commutative. \cite{needed} That is, it is not necessarily the case that
$$M_t(I, J) = M_t(J, I).$$ The commutative property may hold in some cases, but
certainly not all.

\subsection{Notation}

We use Curried notation for function signatures. In the simplest form, we have
$f : A \to B$, where $A$ and $B$ are sets. It is assumed the reader is familiar
with undergraduate-level mathematical notions and notation. We will only explain
where our notation deviates from standard notation.

For functions that take multiple parameters, the standard notation is to use
Cartesian products. For instance, $$f : (A \times B) \to C$$ takes an element
from the set $A$, an element from the set $B$, and produces something from the
set $C$. However, it is often notationally convenient to \emph{Curry} the
type.

\emph{Currying}\footnote{Currying is named after the mathematician Haskell
  Curry. \cite{needed}} is the process of taking a function with type
$$f : (A_1 \times A_2 \times \dots \times A_n) \to B,$$ and turning it into a
chain of functions with type
$$f : A_1 \to (A_2 \to (A_3 \to (\dots \to (A_n \to B))).$$ In the previous
example, we would have $$f : A \to (B \to C).$$ So, $f(a)$, where $a \in A$ is a
function whose type is $B \to C$. The result $f(a, b)$ is a function whose type
is $C$, or, plainly, an element in $C$. If we have
$$g : A \to (B \to (C \to D)),$$ then
$$g(a) : B \to (C \to D), \text{ where } a \in A.$$ $$g(a, b) : C \to D, \text{
  where } a \in A, b \in B.$$ The parentheses are right-associative by default,
so $$h : A \to B \to C \to D \to E$$ is actually just shorthand for
$$h : A \to (B \to (C \to (D \to E))).$$

\printbibliography
\end{document}